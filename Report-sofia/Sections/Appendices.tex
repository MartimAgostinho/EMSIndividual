\pagebreak
\section{Appendices}

Some code was removed to facilitate the reading of the report. The complete code can be found in the attached files in the zip folder submitted.

\subsection{FIR script}

\begin{lstlisting}[language=python, caption=FIR.py]

# -3dB cutoff at 250Hz
# $40 dB attenuation at 300Hz
# linear phase
# sampling rate = 10kHz

# Function to plot the frequency response of a filter
k = 1000
Blue = (0, 101/255, 189/255)
Green = (52/255, 129/255,  65/255)
Orange = (241/255,  89/255,  41/255)


#Description
def myfreqz(filters, w1, w2):
    ...

#Generates C++ Header file for the MCU
def GenHeader(coefs, path=''):
    ...

#Plots Filter window in the time domain
def PlotWindow(FilterWindow):
    ...


# Filter parameters                   # Filter order
cutoff_freq = 270 * 2 * np.pi      # Cutoff frequency (in rad/s)
sampling_rate = 10*k
stopband_frequency = 800 * 2 * np.pi
Amin = -40  # Minimum attenuation in dB the filter windows with this requirement are hamming, hann and blackman

# Pre-warp the cutoff frequency
wc = cutoff_freq
ws = stopband_frequency
Fs = sampling_rate
Ts = 1/Fs
# normalized frequencies
w1 = wc/(Fs*2*np.pi)  # is the normalized cutoff frequency
w2 = ws/(Fs*2*np.pi)  # is the normalized stopband frequency

# Filter order
M = (8 * np.pi) / (w2 - w1)

# M is equal to the first odd integer greater than 8*pi/(w2-w1) odd symmetry -> M is odd -> linear phase
M = math.ceil(M)
if M % 2 == 0:
    M += 1

# Design filters with different windows
windows = ['hann', 'hamming', 'blackman']
filters = {window: sg.firwin(
    M, cutoff=w1, window=window, pass_zero=True, scale=True) for window in windows}
print("M  = "+str(M))
# Plot frequency responses for each filter
myfreqz(filters, w1, w2)

print('w1 =', w1)
print('w2 =', w2)

# print filters
"""
for window in windows:
    coefficients = ", ".join(map(str, filters[window]))
    print(f"{window}\n{coefficients}")
"""
PlotWindow(filters["hamming"])
FreqResponse(filters["hamming"])

# GenHeader(filters["hamming"], path="src/FIR.h")

\end{lstlisting}


\subsection{IIR script}


\begin{lstlisting}[language=python, caption=IIR.py]
k = 1000


def prewrap(wp, Ts):
    return (2/Ts)*np.tan(wp*Ts/2)

#Generates C++ Header file for the MCU
def GenCppHeader(num, den, path=''):
    ...

fp = 300  # is the cutoff frequency
Wn = 2*np.pi*fp  # is the angular frequency
fs = 10*k  # is the sampling frequency
N = 7  # is the order of the filter
omega = prewrap(Wn, 1/fs)  # is the digital frequency
num, den = sg.bessel(N, omega, 'low', analog=True,norm='mag')  # is the transfer function
w, h = sg.freqs(num, den)  # is the frequency response
ampl = 20*np.log10(abs(h))  # is the amplitude in dB
f = w/(2*np.pi)  # is the frequency in Hz

numz, denz = sg.bilinear(num, den, fs)  # is the digital filter
# is the frequency response of the digital filter

wz, hz = sg.freqz(numz, denz)

# mapping of s-plane poles and zeros to z-plane poles and zeros
# s-plane poles and zeros
print('s-plane poles and zeros')
print('num:', num)
print('den:', den)
print('roots of num:', np.roots(num))
print('roots of den:', np.roots(den))

# z-plane poles and zeros
print('z-plane poles and zeros')
print('numz:', numz)
print('denz:', denz)
print('roots of numz:', np.roots(numz))
print('roots of denz:', np.roots(denz))

# -1 is the base 10 logarithm of 0.1 and 4 is the base 10 logarithm of 10000
frequencies = np.logspace(-1, 4, 1000)
wc, hc = sg.freqs(num, den, worN=frequencies)
wz, hz = sg.freqz(numz, denz, worN=frequencies)

#GenCppHeader(numz,denz,path="src/IIR.h")

\end{lstlisting}


\subsection{C++ code}

\begin{lstlisting}[language=C++, caption=Main.cpp]

FIRFilter fir(coefs);
IIRFilter iir(num, den);
CircBuffer<double> FirBuf(fir.WinSize);

void setup(){

    Serial.begin(115200);
    i2sInit();

}

void loop(){
    
    //Read ADC buffer
    esp_err_t e = i2s_read(I2S_NUM_0, &buffer, sizeof(buffer), &bytes_read, portMAX_DELAY);
    int16_t sample_read = bytes_read/2;
   
    //Process Values with IIR Filter
    vector<double> outiir = iir.ProcessSignal(buffer,sample_read);
    double aux;

    if( e == ESP_OK ){
        for (int s = 0; s < sample_read; ++s){
            
            aux = FirBuf.insert(uint12_to_Volt(buffer[s]));
            Serial.print(aux);
            Serial.print(",");
            Serial.print(outiir[s]);

            Serial.print(",");
            FirBuf.next();
            Serial.println(conv(s,FirBuf,fir.coef));
        }
    }
}
\end{lstlisting}

\begin{lstlisting}[language=C++, caption=Filter.h]

template <typename T>
class CircBuffer{
    ...    
//Implements a simple circular buffer

};

class FIRFilter{
    public:
        std::vector<double> coef;
        unsigned int WinSize;


    FIRFilter(const std::vector<double> &Coefs) : coef(Coefs), WinSize(Coefs.size()) {}

};

class IIRFilter{

    private:
        std::vector<double> num;
        std::vector<double> den;
        unsigned int nz; // Number of coef in the numerator
        unsigned int order;
        CircBuffer<double> outBuff;
        CircBuffer<double> sigBuff;

    public:
        unsigned int ord = 0;
    // const unsigned float SampleTime;

    IIRFilter(const std::vector<double> &numerator, const std::vector<double> &denominator)
        : num(numerator),
        den(denominator),
        nz(numerator.size()),
        order(denominator.size()),
        outBuff( order ),
        sigBuff( order ){
        ord = order;
    };

    // The output is Time shifted Ts*order
    // And the output vector has a size = sig.size() - order
    vector<double> ProcessSignal(uint16_t sig[],unsigned int SigSize)
    {

        vector<double> output(order, 0);
        float aux;
        unsigned int CoefPower;

        for (unsigned int SigIndex = order; SigIndex < SigSize; ++SigIndex){
            aux = 0;
            for (unsigned int index = 0; index < nz; ++index){
                CoefPower = nz - index;
                aux += num[index] * uint12_to_Volt(sig[SigIndex - order + CoefPower]);
            }
            for (unsigned int index = 1; index < order; ++index){
                // CoefPower = order - index;
                aux -= den[index] * output[SigIndex - index];
            }
            output.push_back(aux / den[0]);
        }
        output.erase(output.begin(), output.begin() + order);
        return output;
    };
};

double conv( unsigned int pos, CircBuffer<double> sig,vector<double> coef/*, unsigned int SigSize*/ ){
    
    double aux = 0;
    for (unsigned int n = 0; n < coef.size(); ++n){
        //if (pos >= n && pos - n < SigSize){
            aux += sig.get(- n) * coef[n];
        //}
    }
    return aux;
}
\end{lstlisting}



\begin{lstlisting}[language=C++, caption=ADC.h]

//Macro Definitions
... 

double uint12_to_Volt(uint16_t val);
void i2sInit();


void i2sInit(){

    i2s_config_t i2s_config = {
        .mode                   = (i2s_mode_t)(I2S_MODE_MASTER | I2S_MODE_RX | I2S_MODE_ADC_BUILT_IN),
        .sample_rate            = I2S_SAMPLE_RATE,               // The format of the signal using ADC_BUILT_IN
        .bits_per_sample        = I2S_BITS_PER_SAMPLE_16BIT, // is fixed at 12bit, stereo, MSB
        .channel_format         = I2S_CHANNEL_FMT_ONLY_LEFT,
        .communication_format   = i2s_comm_format_t(I2S_COMM_FORMAT_STAND_I2S),//I2S_COMM_FORMAT_I2S_MSB,
        .intr_alloc_flags       = ESP_INTR_FLAG_LEVEL1,
        .dma_buf_count          = 8,
        .dma_buf_len            = I2S_DMA_BUF_LEN,
        .use_apll               = false,
        .tx_desc_auto_clear     = false,
        .fixed_mclk             = 0
    };

    // adc1_config_width(ADC_WIDTH_BIT_12);
    adc1_config_channel_atten(ANALOG_ADC, ADC_ATTEN_DB_0);
    i2s_driver_install(I2S_NUM_0, &i2s_config, 0, NULL);
    i2s_set_adc_mode(ADC_UNIT_1, ANALOG_ADC);
    i2s_adc_enable(I2S_NUM_0);
}

// val must be 12 bit number
double uint12_to_Volt(uint16_t val){

    val = val & 0x0FFF;//uint16 -> uint12
    //Real Voltage in ADC mV
    double adcV = ADC_MIN_VOLT + ((double)val) * ((ADC_MAX_VOLT - ADC_MIN_VOLT) / (double)ADC_MAX);
    return 1e3/*To Micro Volt*/ * (adcV - OFFSET) / GAIN /*AFE Gain and Offset*/;
}
\end{lstlisting}

\subsection{ECG signal generation script}


\begin{lstlisting}[language=python, caption=EcfFourierCoefs.py]
#Script for ECG Signal Fourier Series 

Blue    = (       0, 101/255, 189/255 )
Green   = (  52/255, 129/255,  65/255 )
Orange  = ( 241/255,  89/255,  41/255 )

T   = 5 #duration in seconds
sig = nk.ecg_simulate( duration=T,noise=0.05,heart_rate=60)
Ts  = 1/(10*1000)
N   = len(sig)  # Number of samples
Fs  = 1 / Ts  # Sampling frequency

#-------------Plot Signal-------------#
TimeArray = [ n*Ts for n in range( len(sig) ) ]

fourier_coefficients = np.fft.fft(sig) / N  # Normalized by number of samples

freqs = np.fft.fftfreq(N, Ts)

positive_freqs  = freqs[:N//2]*2*np.pi
phase           = np.angle( fourier_coefficients[:N//2],deg="rad" )
positive_coeffs = abs(fourier_coefficients[:N//2])

#-------------Gen Math Function-------------#

out = ""
for n in range( N//2 ):
    p = ''
    if phase[n] >= 0:
        p = '+'
    if positive_coeffs[n] > 0.0045:
        out += f"{positive_coeffs[n]:.4f}*cos( {positive_freqs[n]}*t {p} {phase[n]:.4f} )+"

print(out[:-1])
\end{lstlisting}
