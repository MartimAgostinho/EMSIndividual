\section{Problem 2}

\subsection{Filter Characterization}

The active filter in figure \ref{fig:P2Circ} as two stages, the first is a High Pass and the second is a Low Pass, making a Band Pass.


\begin{figure}[H]
    \centering
    \includegraphics*[scale = 0.25]{Images/P2Circ.png}
    \caption{Active filter.}
    \label{fig:P2Circ}
\end{figure}

Where the components values are as follow:

\begin{equation}
    \begin{aligned}
        R_1 &= \SI{3.330}{\kilo\ohm}   &;R_2 &= \SI{62.900}{\kilo\ohm}\\
        R_3 &= \SI{120.950}{\kilo\ohm} &;R_4 &= \SI{73.995}{\kilo\ohm}\\
        R_5 &= \SI{18.000}{\kilo\ohm}  &;C_1 &= \SI{0.900}{\nano\farad}\\
        C_2 &= \SI{0.900}{\nano\farad} &;C_3 &= \SI{40.550}{\nano\farad}    
    \end{aligned}
\end{equation}

In order to get the transfer function the following script was made.

\begin{lstlisting}[language=python, caption = Specification Definition]
import scipy as sp 
import scipy.signal as sig
import numpy as np
from sympy import *
import matplotlib.pyplot as plt 
from sympy import symbols, simplify, lambdify, I  # Import lambdify and I (imaginary unit) from SymPy

def lin2dB(val):
    return 20*np.log10(val)

vin, v1, vx, vo, s = symbols("V_{in} V_1 V_x V_o s")

#Passa-baixo

na = 70392
n  = 1e-9
k  = 1000

r1 = 3.33*k
r2 = 62.9*k
r3 = 120.8*k + k*na/(470000) 
r4 = 30*k + k*na/1600
r5 = 18*k
c1 = 0.9*n
c2 = 0.9*n
c3 = 40.7*n - n*na/470000

print(f"R1 = {r1/k:.3f} K   ;R2 = {r2/k:.3f} K")
print(f"R3 = {r3/k:.3f} K   ;R4 = {r4/k:.3f} K")
print(f"R5 = {r5/k:.3f} K   ;C1 = {c1*1e9:.3f} n")
print(f"C2 = {c2*1e9:.3f} n   ;C3 = {c3*1e9:.3f} n")
print("\n")
z1 = 1/(s*c1)
z2 = 1/(s*c2)
z3 = 1/(s*c3)


#Stage 1

vp  = vin* r3/(r3+z3)
if1 = vp/r5
s1  = solve( v1 - if1*r4 - vp,v1)
v1  = simplify(s1[0])

print("\nFirst Stage Output:")
pprint(v1)
print("\nLatex Eq.: \n"+latex(v1))

#Stage 2

v1 = symbols("V_1")
# Vx = V1 - Vr1
saux = solve( vx*z2/(z2+r2) - vo,vx )
#pprint(saux)
vx = saux[0]

s2 = solve( 
    (vx - v1)/r1 + (vx)/(r2+z2) + (vx-vo)/z1
    ,vo)[0]

Fs = simplify( s2 )

print("\nSecond Stage:")
pprint(Fs)
print("\nLatex: \n"+latex(Fs))

Fs = simplify(Fs.subs(v1,s1[0])/vin)
print( "\nFilter Transfer Function: " )
pprint(Fs)
print("\nLatex Eq.: \n"+latex(Fs))

num,den = fraction(Fs)
z       = solve( num,s )
p       = solve( den,s )

print("\n\nFilter Zeros: ")
print(z)
print("Filter Poles: ")
print(p)
print("Frequency:")
for n in p:
    print( f"{abs(n)/(2*np.pi)}" )
 
Fs_numeric = lambdify(s, Fs, 'numpy')

pmax = 6
pmin = 0

freqs  = np.logspace(pmin, pmax, 500)
w = 2 * np.pi * freqs

s_values = 1j * w 
h = Fs_numeric(s_values) 

MaxGain = max(abs(h))
MaxPos  = np.unravel_index(np.argmax(h),h.shape)
print("Max Gain: ",end="")
print(lin2dB(MaxGain))
print("Freq: ",end="")
print( w[MaxPos]/(2*np.pi) )

\end{lstlisting}

The transfer function is shown in equation \ref{eq:P2Tf}.

\begin{equation}
    F_s(s) = \frac{3.094 \cdot 10^{35} s}{\left(2.016 \cdot 10^{13} s + 4.11 \cdot 10^{15}\right) \left(5.095 \cdot 10^{11}s^{2} + 1.79 \cdot 10^{17} s + 3.003 \cdot 10^{21}\right)}
    \label{eq:P2Tf}
\end{equation}

This filter as a zero at $0$, and the poles are:
\begin{itemize}
    \item $p_1 = \SI{32.45}{\hertz}$
    \item $p_2 = \SI{2.81}{\kilo\hertz}$
    \item $p_3 = \SI{53.1}{\kilo\hertz}$
\end{itemize}
This transfer function produces the following Bode, figure \ref{fig:P2BodePy}:

\begin{figure}[H]
    \centering
    \includegraphics*[scale = 0.6]{Images/P2BodePython.png}
    \caption{Filter Bode.}
    \label{fig:P2BodePy}
\end{figure}

The maximum gain is $\SI{14.07}{\decibel}$ at $\SI{299.9}{\hertz}$. The following script, finds the cutoff frequencies.

\begin{lstlisting}[language=python, caption = Cutt Off Frequencies]
ind = np.where(np.isclose(20*np.log10(abs(h)), lin2dB(MaxGain) - 3.01, atol=0.1))
print("CutOff Freqs:")
print(freqs[ind])
\end{lstlisting}

The cuttoff frequencies are $\SI{31.842}{\hertz}$ and $\SI{2824.241}{\hertz}$, this values correspond to the first and second poles.

\subsection{Simulation}

Simulating the circuit in LTSpice the bode is in figure \ref{fig:P2BodeLT}:

\begin{figure}[H]
    \centering
    \includegraphics*[scale = 0.25]{Images/P2BodeLt.png}
    \caption{Filter Bode LTSpice Simulation.}
    \label{fig:P2BodeLT}
\end{figure}

And with the group delay in figure \ref{fig:P2Group}, although it is constant in the beginning and at the end it has a big variation in the pass band.
\begin{figure}[H]
    \centering
    \includegraphics*[scale = 0.25]{Images/P2Group.png}
    \caption{Filter Bode with Group Delay Simulation.}
    \label{fig:P2Group}
\end{figure}

In figure \ref{fig:StepSig} is the voltage supply creating a step. This signal rises from $\SI{0}{\volt}$ to $\SI{0.5}{\volt}$ in $\SI{700}{\nano\second}$.

\begin{figure}[H]
    \centering
    \includegraphics*[scale = 0.5]{Images/StepSig.png}
    \caption{Step signal.}
    \label{fig:StepSig}
\end{figure}


Yielding the step response in figure \ref{fig:StepLT}:

\begin{figure}[H]
    \centering
    \includegraphics*[scale = 0.25]{Images/StepResLT.png}
    \caption{Step response, LTSpice.}
    \label{fig:StepLT}
\end{figure}

The Rise time to $90\%$ is around $\SI{132.78}{\micro\second}$, and because it is a Band Pass it tends to $\SI{0}{\volt}$.

\subsection{Real Opamp Characteristics Impact}


\subsubsection{Gain Bandwith Product (GBW)}

Since the filter has two stages the GBW of each need to be analyzed separately but in the context of the full filter.

For the first stage the Max Gain is $1 + \frac{R_4}{R_5} = \SI{14.17}{\decibel}$ and with a cuttoff Frequency of $\approx \SI{3}{\kilo\hertz}$ therefore the GBW needed in order to not interfere with pass band is $A_d \cdot f_{cuttoff}= \SI{15}{\kilo\hertz}$ hence $10k$ is not enough.

Simulating two circuits of the first stage with $10k$ and $10M$, figure \ref{fig:P2Stg1GBW}, we are able to see that the filter loses gain in the pass band, figure \ref{fig:P2Stg1Bode}.

\begin{figure}[H]
    \centering
    \includegraphics*[scale = 0.4]{Images/P2CircGBWStg1.png}
    \caption{First Stage Circuit.}
    \label{fig:P2Stg1GBW}
\end{figure}

\begin{figure}[H]
    \centering
    \includegraphics*[scale = 0.25]{Images/P2BodeGBWStg1.png}
    \caption{First Stage Bode.}
    \label{fig:P2Stg1Bode}
\end{figure}

For the second stage, the gain is $1$, hence, GBW need is $\SI{3}{\kilo\hertz}$, therefore the Opamp with $10k$ GBW is enough for this stage.

\begin{figure}[H]
    \centering
    \includegraphics*[scale = 0.4]{Images/P2CircGBWStg2.png}
    \caption{Second Stage Circuit.}
    %\label{fig:P2Stg1GBW}
\end{figure}

\begin{figure}[H]
    \centering
    \includegraphics*[scale = 0.25]{Images/P2BodeGBWStg2.png}
    \caption{Second Stage Bode.}
    %\label{fig:P2Stg1Bode}
\end{figure}

