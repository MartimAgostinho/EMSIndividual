\newpage
\section{Problem 6}
\subsection{Principle of operation}

A pressure sensor that uses a Wheatstone bridge operates by converting pressure-induced mechanical deformation into an electrical signal. The key principle involves strain gauges that change their resistance when stretched or compressed, as seen in figure \ref{fig:pressSens}, were $R_1 = R_3 = R-\Delta R$ and $R_2 = R_4 = R + \Delta R$. When pressure is applied, the value $\Delta R$ changes, making the output value, $V_+ -V_- = V_{out}$, change accordingly to pressure.

\begin{figure}[H]
    \centering
    \includegraphics*[scale = 0.3]{Images/pressureWheatstone.png}
    \caption{Pressure Sensor Wheatstone circuit.}
    \label{fig:pressSens}
\end{figure}

The output Voltage is as shown in the equation \ref{eq:bridgeOut}:
\begin{equation}
    \label{eq:bridgeOut}
    V_{out} = V_{in}\frac{\Delta R}{R}
\end{equation}

Since $\Delta R$ is proportional to pressure the relation of $V_{out}$ and pressure is shown in equation \ref{eq:pressVal}:
\begin{equation}
    \label{eq:pressVal}
    V_{out} = V_{in}FS
\end{equation}

Where $F$ is the pressure and $S$ is the sensor sensitivity in $\frac{\si{\m\V}}{\si{V}}$, this value is specified by manufacture.

\subsection{Smart Pressure Controlled Compressor}

In applications were consistent pressure control, safety and efficiency is essential, a compressor that is able to output a consistent pressure, through a feedback system, might be the solution. This kind of compressor can be useful to fill tires to a consistent value, or even maintain a vacuum line pressure and change it remotely. 

The system block diagram is shown in figure \ref{fig:BlockDiagramP6}. A compressor is responsible to ensure pressure, the electrical valve controls how much pressure there is at the output and the pressure sensor measures the pressure at the output. With an ESP32 as the MCU the system is able to preform a feedback loop to control the pressure set by the user, via HomeAssistant or a Potentiometer for manual adjustment.

\begin{figure}[H]
    \centering
    \includegraphics*[scale = 0.5]{Images/P6Block.png}
    \caption{Compressor System Block Diagram.}
    \label{fig:BlockDiagramP6}
\end{figure}

\subsection{Design}
\subsubsection{Components}
For the design it is important to define the pressure range, for a multipurpose compressor the pressure sensor maximum pressure must be higher than 4 bar, therefor the sensor chosen was the HSCDAND015PG2A3 and for the valve the ASCO Series 209 Proportional Valve. It is important to note that the for valve a logic level-shifter might be needed for communication.

In order to power the system the ESP32 needs a $\SI{5}{\V}$, the sensor have a typical power supply of $\SI{3.3}{\V}$ hence can be powered by the ESP32 and the valve is power with $\SI{12}{\V}$. Because at least $\SI{12}{\V}$ and $\SI{5}{\V}$ are needed, some kind of DC-DC converter is needed. This can be achieved with a buck converter

\subsubsection{System}

$\rightarrow$ FeedBack control Loop:

For the pressure control, the ESP32 will work as PID controller, since this is the simplest and most reliable controller implementation. The PID parameters shall be configurable through the HomeAssistant GUI, in order to calibrate the system.

The sensor is read through I2S and the valve is adjusted trough the PWM signal.

$\rightarrow$ User Input:

The IOT is handled through the HomeAssistant. HomeAssistant is an Open Source IOT integration platform. The ESP32 is able to receive new values of pressure through MQTT set in the HomeAssistant UI. 

Because reliability is important a potentiometer is added for manual pressure control, since in an event were there is no internet or simply there is no good connection to the HomeAssistant server, this way the system is not rendered useless.

$\rightarrow$ Visualization:

For visualization there would be a simple display, to show the current pressure. And a panel in the HomeAssistant UI for remote Visualization.
