\section{Problem 5}

This problem goal is to design a system that will turn on and off an air fan, depending on the temperature.

\subsection{Specifications}

This are the specifications for this particular problem.

The temperature intervals for turning on and off the system:
\begin{itemize}
        
    \item $T_1 = 42~C^\circ   =  315.15 ~K$
    \item $T_2 = 42.5~C^\circ =  315.65 ~K$

\end{itemize}

In order to sense the temperature, this system uses an NTC. To correlate its resistance with temperature the beta model was used. 

\begin{equation}
    R = R_0 e^{\beta\left ( \frac{1}{T} - \frac{1}{T_0 } \right )}
\end{equation}

From the datasheet \textsuperscript{\cite{NTC_datasheet}} :

\begin{itemize}
    \item $\beta = 3988$
    \item $R(25^{\circ}) = 5k\Omega$ 
\end{itemize}

Therefore: 
\begin{equation}
    R = 5K\cdot e^{ 3988\cdot \left ( \frac{1}{T} - \frac{1}{298.15 } \right )}
\end{equation}

Hence: 
\begin{equation}
    \begin{aligned}   
        R(T_1) = \SI{2.43}{\kilo\ohm} \\
        R(T_2) = \SI{2.38}{\kilo\ohm}
    \end{aligned}
\end{equation}

\subsection{Design}
\subsubsection{Comparator With Hysteresis}

For the comparator with hysteresis, as suggested, the circuit in figure \ref{fig:ComparatorCirc}, was considered, but cannot be used since it inverts the output.

\begin{figure}[H]
    \centering
    \includegraphics*[scale = 0.5]{Images/SingleSupplySchimidtTrigger.png}
    \caption{Comparator with hysteresis circuit \textsuperscript{\cite{NTC_datasheet}}}
    \label{fig:ComparatorCirc}
\end{figure}

In order to make this circuit not inverting, without switching the NTC with $R_{1}$. The circuit in figure \ref{fig:NTCCirc} was used.

\begin{figure}[H]
    \centering
    \includegraphics*[scale = 0.5]{Images/NTCCirc.png}
    \caption{NTC with non-inverting comparator}
    \label{fig:NTCCirc}
\end{figure}

In this circuit the NTC alongside with $R_1$ work as a voltage divider, creating the voltage $V_{in}$, and for the voltage $V_{ref}$, $R_2$ and $R_3$ also make a voltage divider. 
For this circuit to work $R_5$ and $R_f$ need to be considerably bigger than $NTC$ and $R_2$, or a buffer could also be used.

\begin{figure}[H]
    \centering
    \includegraphics*[scale = 0.5]{Images/HysteresisWindow.png}
    \caption{Hysteresis window \textsuperscript{\cite{Individual_statement}}}
    \label{fig:HysterWindow}
\end{figure}

Considering the hysteresis window, figure \ref{fig:HysterWindow} and $R_1 = 5K$.

\begin{equation}
    \begin{aligned}
        V_H &= 5\frac{R_1}{R_1 + NTC(T_2)} = 3.388 \si{\V}\\
        V_L &= 5\frac{R_1}{R_1 + NTC(T_1)} = 3.365 \si{\V}
    \end{aligned}
\end{equation}


When $V_{out} = HIGH$:

\begin{equation}
    V_+ = V_{CC}-\left (V_{CC}-V_L\right )\frac{R_5}{R_5 + R_f}
\end{equation}

When $V_{out} = LOW$:

\begin{equation}
    V_+ = V_H\frac{R_f}{R_f + R_5}
    \label{eq:VpLow}
\end{equation}
Hence:
\begin{equation}
    V_{CC}-\left (V_{CC}-V_L\right )\frac{R_5}{R_5 + R_f} = V_H\frac{R_f}{R_f + R_5}
    \label{eq:solR5}
\end{equation}

Since $R_f$ needs to be $>>5K$ and since the output is either high or low thermal noise should not be problem, $R_f$ was set as $\SI{1}{\mega\Omega}$. 

Therefore, solving equation \ref{eq:solR5}, $R_5 = \SI{4.6}{\kilo\ohm}$ and $V_{ref} = 3.37\si{\V}$.
\subsection{Opamp Offset}

The Opamp offset is represented as a voltage supply at the non-inverting input. Meaning the  $V_{+} = V_{ref} + V_{offset}$.

For the Opamp used, the MCP6004, the input offset is $\pm 4.5\si{\milli\volt}$ \textsuperscript{\cite{MCP6001_datasheet}}. 

From rewriting equation \ref{eq:VpLow}, with the Opamp offset:

\begin{equation}
    V_H = \left ( V_+ + V_{offset}\right )\frac{R_f +R_5}{R_f}
\end{equation}
 
Hence, for the comparator switch from LOW to HIGH $V_{in} = 3.388 + 4.52\cdot 10^{-3} = 3.393$, this represents a difference from the goal temperature of $\SI{0.144}{\kelvin}$, which for this application is not significant.

\subsection{Relay Circuit}
In order to turn the fan on and off, a relay needs to be powered by a transistor, since the Opamp does not output enough current, to turn the relay on and off. The circuit in figure \ref{fig:RelayCirc} was considered. The diode is added in order to protect the relay from voltage spikes cause by the sudden changes in current.

\begin{figure}[H]
    \centering
    \includegraphics*[scale = 0.2]{Images/RelayCirc.png}
    \caption{Relay Circuit.}
    \label{fig:RelayCirc}
\end{figure}

\begin{equation}
    \begin{aligned}
        I_C &= \beta \cdot I_B\\
        I_B &= \frac{V_o - V_{BE}}{R}\\ 
        \Rightarrow R &= \beta \frac{V_o - V_{BE}}{I_C}
    \end{aligned}
\end{equation}

From the NPN transistor 2N2222 datasheet \textsuperscript{\cite{2N2222_datasheet}}, the following values were taken:
\begin{itemize}
    \item $V_{BE} = $
    \item $\beta = $
\end{itemize}
And from the relay datasheet \textsuperscript{\cite{ST2-DC5V-F_datasheet}}, $I_C = I_{Relay} = $.

Hence $R = $.

\subsubsection{Comparator Simulation}

In order to test the design the circuit in figure \ref{fig:TestCirc} was tested in LTSpice, the value of NTC varies from $2k\Omega$ to $4k\Omega$ and back to $2k\Omega$.

\begin{figure}[H]
    \centering
    \includegraphics*[scale = 0.4]{Images/CompSimCirc.png}
    \caption{Simulation test circuit.}
    \label{fig:TestCirc}
\end{figure}

\begin{figure}[H]
    \centering
    \includegraphics*[scale = 0.25]{Images/CompSimRes.png}
    \caption{Simulation test results.}
    \label{fig:TestRes}
\end{figure}

As shown in figure \ref{fig:TestRes}, the output starts at HIGH and therefore the $V_{in}$ value to change the output to LOW needs to be $\approx 3.365V$ and in the simulation the value is $3.37V$. As for the change from LOW back to HIGH the designed value was $\approx 3.388V$ and in the simulation it changed at $3.39V$, this confirmed the effectiveness of the comparator circuit.


