\section{Problem 1}

\subsection{Voltage Gain}

Considering an ideal OPAMP  working on the linear region, the following assumptions were made.

\begin{itemize}
    \item $Z_{in} = +\infty$
    \item $Z_{out} = 0$
    \item $A_d = +\infty$
    \item $V_+ = V_-$
\end{itemize}

In order to obtain the value of $V_{out}(V_{in})$, it is necessary to get the circuit equations.

\begin{equation}
    \begin{cases}
    
        i_{t}   = \frac{V_-}{R_1}\\
        V_x     = V_- + R_2 \cdot i_t\\ 
        V_{out} = V_x + R_3\cdot i_3\\
        i_3     = i_t + i_4 = i_t + \frac{V_x}{R_4} 

    
    \end{cases}
\end{equation}

Where $i_t$ is the current passing through $R_1$ and $R_2$, $V_x$ is voltage between $R_3$ terminals.
This results in the following equation.

\begin{equation}
    V_{out} = \frac{ R_{1} R_{4} + R_{2} R_{4} + R_{3} \left(R_{1} + R_{2} + R_{4}\right)}{R_{1} R_{4}}\cdot V_{i}
\end{equation}

\subsection{ Input Current Bias }

In order to calculate the impact of the current bias, a current supply is placed in parallel on the input terminals of the OPAMP. As shown in the figure 

\begin{figure}[H]
    \centering
    \includegraphics*[scale = 0.25]{Images/Ex1Bias.png}
    \caption{Circuit with input bias current}
    \label{Ex1Bias}
\end{figure}

Using superposition, to evaluate the current effect on the output, the following system of equations is obtained.


\begin{equation}
    \begin{cases}
    
        V_- = V_+ = 0 \\
        I_{R_1} = 0\\
        \frac{V_x}{R_4} + \frac{V_x - V_{out}}{R_3} + I_{bias-}\\
        V_{out} = I_{bias-} - \frac{V_x - V_{out}}{R_3}

    \end{cases}
\end{equation}

Solving:
\begin{equation}
    V_{out} = \frac{I_{BIAS-} \left(R_{2} R_{3} + R_{2} R_{4} - R_{3}^{2} R_{4}\right)}{R_{3}^{2} + R_{3} + R_{4}}
\end{equation}

\subsection{Opamp Offset Voltage}

The Opamp offset voltage is represented through a series voltage supply in the non inverting input, therefore through superposition $V_+ = V_{offet}$, hence:

\begin{equation}
     V_{out} = \frac{ R_{1} R_{4} + R_{2} R_{4} + R_{3} \left(R_{1} + R_{2} + R_{4}\right)}{R_{1} R_{4}}\cdot V_{offset}
\end{equation}


\subsection{Voltage Input Range}

Taking into account the total gain of the circuit $A_d$ and the offset voltage $V{offset}$ of the OpAmp, the input voltage range for this circuit is:

\begin{equation}
   V_{in} \in \left [ \frac{V_{Sat-}}{A_d} - V_{offset}, \frac{V_{Sat+}}{A_d} - V_{offset} \right ]
\end{equation}

\subsection{Output Noise}

The output noise can be calculated by placing a voltage supply in series with any resistor in the circuit, where this supply RMS value is $4K\cdot T\cdot R\cdot Bw$.

In order to analyse each component noise, the theorem of superposition is use, yielding the system of equation \ref{eq:NoiseP1}.

\begin{equation}
    \begin{cases}
        V_{O1n} &= V_{R1_nrms} \frac{- R_{2} R_{3} - R_{2} R_{4} - R_{3} R_{4}}{R_{1} R_{4}} \\
        V_{O2n} &= V_{R2_nrms} \left(1 + \frac{R_{3}}{R_{4}}\right)  \\
        V_{O3n} &= V_{R3_nrms}\\
        V_{O4n} &= V_{R4_nrms} \left(1 + \frac{R_{3}}{R_{4}} \right)\\
        V_{O5n} &= V_{In_nrms}\frac{ R_{1} R_{4} + R_{2} R_{4} + R_{3} \left(R_{1} + R_{2} + R_{4}\right)}{R_{1} R_{4}}\\
        V_{O6n} &= I_{Inn_nrms} + \frac{R_{1} \left(R_{2} R_{3} + R_{2} R_{4} + R_{3} R_{4}\right)}{- R_{1} R_{4} + R_{2} R_{3} + R_{2} R_{4} + R_{3} R_{4}} \\
        V_{O7n} &= V_{nOpAmp}
    \end{cases}
    \label{eq:NoiseP1}
\end{equation}