\section{Problem 5}

\subsection{Considerations}

\begin{itemize}
        
    \item $T1 = 42~C^\circ   =  315.15 ~K$
    \item $T2 = 42.5~C^\circ =  315.65 ~K$

\end{itemize}
\subsection{NTC}

Using the beta model:

\begin{equation}
    R = R_0 e^{\beta\left ( \frac{1}{T} - \frac{1}{T_0 } \right )}
\end{equation}


From the datasheet :
\begin{itemize}
    \item $\beta = 3988$
    \item $R(25^{\circ}) = 5k\Omega$ 
\end{itemize}

Therefore: 
\begin{equation}
    R = 5K\cdot e^{ 3988\cdot \left ( \frac{1}{T} - \frac{1}{298.15 } \right )}
\end{equation}

Hence: 
R(T1) = 
R(T2) = 

%\subsection{Comparator With Hysteresis}



\begin{figure}[H]
    \centering
    \includegraphics*[scale = 0.5]{Images/SingleSupplySchimidtTrigger.png}
    \caption{Comparator with hysteresis circuit \textsubscript{\cite{NTC_datasheet}}}
    \label{fig:ComparatorCirc}
\end{figure}

For circuit dimensioning the following equations were used\textsuperscript{\cite{TI-Comparator-Hysteresis}}.

\begin{equation}
    \begin{cases}
        \frac{R_f}{R_1} = \frac{V_L}{V_H - V_L}\\
        \frac{R_2}{R_1} = \frac{V_L}{V_{CC}-V_{H}}
    \end{cases}
\end{equation}

But since this circuit is inverting.

\begin{figure}[H]
    \centering
    \includegraphics*[scale = 0.5]{Images/NTCCirc.png}
    \caption{NTC with non-inverting comparator}
    \label{fig:NTCCirc}
\end{figure}

In this circuit $V_{in} = V_{CC}\cdot\frac{R_3}{R_3+R_1}$, $R_1 = R_{NTC}$.