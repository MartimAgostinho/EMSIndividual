\section{Problem 1}

\subsection{Voltage Gain}

Considering an ideal OPAMP  working on the linear region, the following assumptions were made.

\begin{itemize}
    \item $Z_{in} = +\infty$
    \item $Z_{out} = 0$
    \item $A_d = +\infty$
    \item $V_+ = V_-$
\end{itemize}

In order to obtain the value of $V_{out}(V_{in})$, it is necessary to get the circuit equations.

\begin{equation}
    \begin{cases}
    
        i_{t}   = \frac{V_-}{R_1}\\
        V_x     = V_- + R_2 \cdot i_t\\ 
        V_{out} = V_x + R_3\cdot i_3\\
        i_3     = i_t + i_4 = i_t + \frac{V_x}{R_4} 

    
    \end{cases}
\end{equation}

Where $i_t$ is the current passing through $R_1$ and $R_2$, $V_x$ is voltage between $R_3$ terminals.
This results in the following equation.

\begin{equation}
    V_{out} = \frac{ R_{1} R_{4} + R_{2} R_{4} + R_{3} \left(R_{1} + R_{2} + R_{4}\right)}{R_{1} R_{4}}\cdot V_{i}
\end{equation}

\subsection{ Input Current Bias }

In order to calculate the impact of the current bias, a current supply is placed in parallel on the input terminals of the OPAMP. As shown in the figure 

\begin{figure}[H]
    \centering
    \includegraphics*[scale = 0.25]{Images/Ex1Bias.png}
    \caption{Circuit with input bias current}
    \label{Ex1Bias}
\end{figure}

Using superposition, to evaluate the current effect on the output, the following system of equations is obtained.


\begin{equation}
    \begin{cases}
    
        V_- = V_+ = 0 \\
        I_{R_1} = 0\\
        \frac{V_x}{R_4} + \frac{V_x - V_{out}}/R_3 + I_{bias-}\\
        V_{out} = I_{bias-} - \frac{V_x - V_{out}}{R_3}

    \end{cases}
\end{equation}
So, the relation between the output voltage and the current bias is expressed in equation \ref{Eq:voutibias}.
\begin{equation}
    V_{out} = \frac{I_{BIAS-} \left(R_{2} R_{3} + R_{2} R_{4} - R_{3}^{2} R_{4}\right)}{R_{3}^{2} + R_{3} + R_{4}}
    \label{Eq:voutibias}
\end{equation}

\subsection{Voltage offset}

To calculate the impact of the voltage offset of the OPAMP, a voltage supply is added in the negative input terminal of the OPAMP, as shown in Figure \ref{Ex1offset}.

\begin{figure}[H]
    \centering
    \includegraphics*[scale = 0.6]{Images/Voffsim.png}
    \caption{Circuit with input voltage offset}
    \label{Ex1offset}
\end{figure}

The $V_{Off}$ will only affect the DC value of the output signal, this value is obtained by the equation \ref{eq:offset}.

\begin{equation}
    V_{outDC} = A_d \cdot (V_{Off} + V_{inDC})
    \label{eq:offset}
\end{equation}
\subsection{Voltage Input Range}

Considering the total gain of the circuit $A_d$ and the $V_{Off}$ of the OpAmp, the maximum input voltage for this circuit will be:

\begin{equation}
   V_{in} \in [\frac{V_{Sat-} + V_{Off}}{A_d}, \frac{V_{Sat+} + V_{Off}}{A_d} ]
\end{equation}

\subsection{Output Noise}

The output voltage noise can be calculated using a process similar to the superposition theorem, where using the RMS value of the noise generated by all components, the final value can be acquired. 

To calculate the noise of witch component, for the resistors, this component must be replaced by a noiseless resistance and a random voltage source. The circuit used for the noise calculation is shown in figure \ref{Ex1noise}

\begin{figure}[H]
    \centering
    \includegraphics*[scale = 0.35]{Images/EX1_noise.png}
    \caption{Circuit with noise components}
    \label{Ex1noise}
\end{figure}

For each noise source, the output noise voltage will be:

\begin{equation}
    \begin{aligned}
        V_{O1nrms} &= \frac{V_{R1_nrms} \cdot \left(- R_{2} R_{3} - R_{2} R_{4} - R_{3} R_{4}\right)}{R_{1} R_{4}} \\
        V_{O2nrms} &= V_{R2_nrms} \cdot \left(1 + \frac{R_{3}}{R_{4}}\right)  \\
        V_{O3nrms} &= V_{R3_nrms}\\
        V_{O4nrms} &= V_{R4_nrms} \left(1 + \frac{R_{3}}{R_{4}} \right)\\
        V_{O5nrms} &= \frac{V_{In_nrms} \left(R_{1} R_{4} + R_{2} R_{4} + R_{3} \left(R_{1} + R_{2} + R_{4}\right)\right)}{R_{1} R_{4}}\\
        V_{O6nrms} &= I_{Inn_nrms} + \frac{R_{1} \left(R_{2} R_{3} + R_{2} R_{4} + R_{3} R_{4}\right)}{- R_{1} R_{4} + R_{2} R_{3} + R_{2} R_{4} + R_{3} R_{4}} \\
        V_{O7nrms} &= V_{nOpAmp}
    \end{aligned}
\end{equation}

Because in noise, only the power of every contribution can be summed, the output noise power will be:

\begin{equation}
    \overline{V_{On}^2} = \sum_{k = 1}^{7} V_{Oknrms}^2
\end{equation}